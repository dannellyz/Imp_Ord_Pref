This paper investigates the the possibility of uncovering implied ordinal preferences when provided incomplete ordinal preferences for one side of a matching market. The work of Gale and Shapely's \cite{gale_shapely} provided the first algorithm for solving the stable marriage and attempted the more general college admissions problems (verified as different problems by Roth \cite{canmp}). Much of the work in this field focuses on stability, which is not necessarily a requirement in systems where market participants are compelled \cite{incentives}. In situations of compulsion, the preference $u$ (un-assigned) is not an option. This arises in legally compulsory assignment situations such as jobs in the military or secondary school enrollment. 

In these situation, complete ordinal preferences are not always provided. This can be due to a lack of time, lack of knowledge about preferences, excess of options, or some other system deficiency that does not encourage/enable participants to express preferences on all matching options. 

This paper proposes a system where, given at least one preference for all participants on one side of the market, preferences may be supposed at a higher accuracy than random preference assumption.

The code to demonstrate the matching algorithms, optimization, and preference-based metrics can be found in Ian Shaw's Github Repository. \footnote{\url{https://github.com/ieshaw/Imp_Ord_Pref}}
